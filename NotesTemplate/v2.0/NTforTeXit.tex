% Made by: Aathreya Kadambi
% This is actually my first package :D
% Important note: Many of these commands that I made were based off or taken from code snippets that I found on stack exchange.
%
% This is a copy of NotesTemplate.sty for which you can directly copy the code and place it at the top of your .tex file. It was made for use with TeXit, a bot on Discord
%	VERSION 2.0
%
% I basically took this structure off of https://tex.stackexchange.com/questions/8750/make-your-own-sty-files.

% PACKAGES
\usepackage[dvipsnames]{xcolor}
\usepackage{amsmath}
\usepackage{amsfonts}
\usepackage{amssymb}
\usepackage{ulem}
\usepackage{url}
\usepackage{tikz}
\usepackage{hyperref}
\usepackage{wrapfig}
\usepackage{indentfirst}
\usepackage{array}
\usepackage{booktabs}
\usepackage{ragged2e}
\usepackage{mathtools}
\usepackage{enumitem}
\usepackage[T1]{fontenc}
\usepackage[a4paper, margin=1in]{geometry}
\usepackage{tcolorbox}


% PACKAGES WITH TIKZ
\usepackage{tkz-euclide}
\usetikzlibrary{calc,through,intersections,positioning}

% USER DEFINED VARIABLES
\newcommand{\R}{\mathbb{R}}
\newcommand{\N}{\mathbb{N}}
\newcommand{\Q}{\mathbb{Q}}
\newcommand{\Primes}{\mathbb{P}}
\newcommand{\C}{\mathbb{C}}

\definecolor{dr}{RGB}{175, 0, 0}
\newcommand{\imp}[1]{\textcolor{dr}{#1}}

\newcommand{\myeq}[1]{\stackrel{\mathclap{\normalfont\mbox{\tiny{$#1$}}}}{=}} % For projective geometry

\newcommand{\QED}{\hfill\null Q.E.D.}
\newcommand{\EL}{\hfill\null End Lemma.}

% TIKZ Shortcuts
\newcommand{\Tgrid}[4]{\draw[step=1cm, gray, very thin, ->] (#1, #2) grid (#3, #4); \draw (#1, 0) -- (#3, 0); \draw (0, #2) -- (0, #4);}
\newcommand{\Tplane}[4]{\draw[<->] (#1, 0) -- (#3, 0); \draw[<->] (0, #2) -- (0, #4);}

\renewcommand{\theequation}{\thechapter.\arabic{equation}}

\newcommand{\abisect}[6][] % Source: https://tex.stackexchange.com/questions/233246/create-angle-bisectors-in-tikz
{
	\path[#1] let
	\p1 = ($(#3)!1cm!(#2)$),
	\p2 = ($(#3)!1cm!(#4)$),
	\p3 = ($(\p1) + (\p2) - (#3)$)
	in
	($(#3)!#6!(\p3)$) -- ($(\p3)!#5!(#3)$) ;
}

% COLORFUL BOXES
\definecolor{NiceBlue}{HTML}{00FFFF}

\newtcolorbox{Strat}[1][]{
	frame hidden, 
	borderline north = {1.6pt}{0pt}{NiceBlue},
	borderline south = {1.6pt}{0pt}{NiceBlue},
	interior style = {top color=Aquamarine!25, bottom color=Aquamarine!15},
	#1
}

\newtcolorbox{Skill}[1][]{
	frame hidden, 
	borderline north = {1.6pt}{0pt}{red},
	borderline south = {1.6pt}{0pt}{red},
	interior style = {top color=red!15, bottom color=red!10},
	#1
}

\newtcolorbox{Theorem}[1][]{
	frame hidden, 
	borderline north = {1.6pt}{0pt}{Fuchsia},
	borderline south = {1.6pt}{0pt}{Fuchsia},
	interior style = {top color=Fuchsia!15, bottom color=Fuchsia!10},
	#1
}

\newtcolorbox{Interesting}[1][]{
	frame hidden, 
	borderline west = {1.6pt}{0pt}{green},
	interior style = {top color=green!15, bottom color=green!10},
	#1
}

\newtcolorbox{Problem}[1][]{
	frame hidden, 
	borderline west = {1.6pt}{0.5pt}{black},
	interior style = {top color=black!15, bottom color=black!15},
	#1
}

\newtcolorbox{Remark}[1][]{
	frame hidden, 
	borderline west = {1.6pt}{0.5pt}{Yellow},
	interior style = {left color=Yellow!35, right color=white},
	#1
}

% Legacy Boxes
\newsavebox{\labox}
\newenvironment{StratL}
{
	\newcommand{\colboxcolor}{E0F6FF}
	\newcommand{\bordercolor}{00FFFF}
	\begin{lrbox}{\labox}
		\begin{minipage}{\dimexpr\columnwidth-2\fboxsep\relax}
		}
		{
		\end{minipage}
	\end{lrbox}
	\begin{center}
		{\setlength{\fboxsep}{2.8pt}\setlength{\fboxrule}{1.8pt}
			\fcolorbox[HTML]{\bordercolor}{\colboxcolor}{\usebox{\labox}}}
	\end{center}
}

\newenvironment{SkillL}
{
	\newcommand{\colboxcolor}{FFE0E0}
	\newcommand{\bordercolor}{FF0000}
	\begin{lrbox}{\labox}
		\begin{minipage}{\dimexpr\columnwidth-2\fboxsep\relax}
		}
		{
		\end{minipage}
	\end{lrbox}
	\begin{center}
		{\setlength{\fboxsep}{2.8pt}\setlength{\fboxrule}{1.8pt}
			\fcolorbox[HTML]{\bordercolor}{\colboxcolor}{\usebox{\labox}}}
	\end{center}
}

\newenvironment{InterestingL}
{
	\newcommand{\colboxcolor}{E0FFE0}
	\newcommand{\bordercolor}{00FF00}
	\begin{lrbox}{\labox}
		\begin{minipage}{\dimexpr\columnwidth-2\fboxsep\relax}
		}
		{
		\end{minipage}
	\end{lrbox}
	\begin{center}
		{\setlength{\fboxsep}{2.8pt}\setlength{\fboxrule}{1.8pt}
			\fcolorbox[HTML]{\bordercolor}{\colboxcolor}{\usebox{\labox}}}
	\end{center}
}

\newenvironment{ProblemL}
{
	\newcommand{\colboxcolor}{EEEEEE}
	\newcommand{\bordercolor}{777777}
	\begin{lrbox}{\labox}
		\begin{minipage}{\dimexpr\columnwidth-2\fboxsep\relax}
		}
		{
		\end{minipage}
	\end{lrbox}
	\begin{center}
		{\setlength{\fboxsep}{2.8pt}\setlength{\fboxrule}{1.8pt}
			\fcolorbox[HTML]{\bordercolor}{\colboxcolor}{\usebox{\labox}}}
	\end{center}
}

\newenvironment{ColorBoxL}[2]
{
	\newcommand{\colboxcolor}{#1}
	\newcommand{\bordercolor}{#2}
	\begin{lrbox}{\labox}
		\begin{minipage}{\dimexpr\columnwidth-2\fboxsep\relax}
		}
		{
		\end{minipage}
	\end{lrbox}
	\begin{center}
		{\setlength{\fboxsep}{2.8pt}\setlength{\fboxrule}{1.8pt}
			\fcolorbox[HTML]{\bordercolor}{\colboxcolor}{\usebox{\labox}}}
	\end{center}
}

% The following snippet makes it possible to make colorful bullet points
\newlist{coloritemize}{itemize}{1}
\setlist[coloritemize]{label=\textcolor{itemizecolor}{\textbullet}}
\colorlet{itemizecolor}{.} % This is the command for changing color

% The following snippets make it possible to use a cooler font that I particularly like
\newenvironment{KoolFont}{\fontfamily{lmss}\selectfont}{}

\newcommand{\KF}[1]{{\fontfamily{lmss}\selectfont #1}}

% The following snippet makes the augmented matrices work correctly with [ccc|c] thing
\makeatletter
\renewcommand*\env@matrix[1][*\c@MaxMatrixCols c]{%
  \hskip -\arraycolsep
  \let\@ifnextchar\new@ifnextchar
  \array{#1}}
\makeatother

%%
%% End of file `NTforTeXit.tex'.
